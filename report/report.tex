\documentclass[11pt]{report}
\usepackage[italian]{babel}
\usepackage[utf8]{inputenc}
\usepackage[T1]{fontenc}
\usepackage[nomarginpar]{geometry}

\usepackage[pdftex,
			pdfauthor={Eugenio Severi, Stefano Belli},
			pdftitle={Progetto ASW - id3king-js},
			pdfsubject={Relazione progetto Applicazioni e Servizi Web},
			pdfkeywords={id3king id3king-js ASW},
			pdfproducer={Latex with hyperref},
			pdfcreator={pdflatex}]{hyperref}
\pagestyle{plain}

\begin{document}
\title{Progetto di Applicazioni e Servizi Web\\id3king-js}
\author{Eugenio Severi, Stefano Belli}
\date{A.A. 2016-2017}
\begin{titlepage}
	\maketitle
\end{titlepage}

\setcounter{chapter}{1}
\section{Introduzione}
Il sito "id3king.it" è un progetto amatoriale di un gruppo forlivese che raccoglie percorsi escursionistici di trekking in varie zone d'Italia.
Ogni volta che il gruppo fa un'escursione, ne pubblica sul sito i dettagli: data dell'uscita, luogo, durata, difficoltà, distanza e dislivello percorsi, mappe, dettagli del percorso e foto.
\\Tuttavia, l'interfaccia grafica del sito è poco pratica, in quanto risulta antiquata, priva di funzionalità di ricerca e, osservando il modo in cui i dati vengono proposti, pare che non siano normalizzati e memorizzati su un database in maniera opportuna.
Per questo motivo, ottenuto il consenso da parte degli amministratori del sito, si è deciso di progettare e sviluppare una web app che esegue lo scraping dell'originale e propone i medesimi dati in maniera maggiormente fruibile e aggiungendo ulteriori funzionalità, quali la ricerca basata su parametri e la creazione di profili utente per memorizzare dati di interesse.
\\Sarà quindi necessario realizzare un'opportuna architettura server, un back-end e un front-end. Il sistema dovrà rispettare requisiti di estensibilità, reattività e sicurezza.
% TODO: diagramma dei casi d'uso

\section{Architettura}
Trattandosi di un'applicazione di tipo client-server, l'architettura è composta dai seguenti componenti:
\begin{itemize}
	\item un insieme di client che accedono al servizio;
	\item un server web a cui i client si collegano;
	\item un database relazionale per l'archiviazione dei dati.
\end{itemize}
Opzionalmente è possibile rendere l'applicazione scalabile orizzontalmente e resistente ai guasti tramite più server web e load balancer.
Analogamente, è possibile utilizzare non un singolo DBMS, ma una configurazione cluster.
\\Ai fini di questo progetto si è scelto di utilizzare un unico server cloud di tipo \textit{IaaS} (Infrastructure as a Service), fornito da un provider, sul quale saranno installati tutti i servizi necessari al funzionamento dell'applicazione.
\\In base ai requisiti precedentemente definiti, come framework di sviluppo si è valutato di usare \textit{Node.js}, che prevede l'utilizzo di tecnologie web sia client-side che server-side e consente di utilizzare un modello di comunicazione event-driven, particolarmente utile in un'applicazione di rete.
Saranno presenti interfacce REST, che prevedono una comunicazione \textit{stateless} (ovvero non necessita della memorizzazione di alcun contesto client sul server), consentono di implementare facilmente un'architettura scalabile a livelli con nodi di commutazione intermedi e sono universalmente riconosciute e supportate, essendo basate tu HTTP+TCP+IP.
% TODO: diagramma di deployment

\section{Sicurezza}
Data la natura di rete dell'applicazione e poiché verranno trattati anche alcuni dati personali degli utenti, è opportuno considerare la sicurezza già in fase di progettazione, onde evitare lacune successive difficili da individuare e rimuovere:
\begin{itemize}
	\item tutte le comunicazioni di rete devono essere crittografate e autenticate con SSL/TLS per prevenire intercettazioni e attacchi \textit{man-in-the-middle};
	\item tutti gli input degli utenti devono essere validati per impedire lo sfruttamento di \textit{SQL injection} e \textit{buffer overflow}.
\end{itemize}
Le credenziali degli utenti dovranno inoltre essere memorizzate sotto forma di hash della password e con \textit{salt}, onde rendere difficoltoso per un attaccante decifrare le password.
A tale scopo si è scelta la funzione \textit{bcrypt}.

\section{Back-end}
Verrà ora descritto il funzionamento del back-end.
%TODO: diagramma delle classi e/o di sequenza
%TODO: Dati e database, controller (pattern MVC), interazioni con il db, scraper.

\section{Front-end}
%TODO: Angular in generale, pattern MVC (e altri?), descrizione interfaccia grafica (spiegazione funzioni principali con qualche screenshot).

\section{Conclusioni}


\end{document}
\endinput