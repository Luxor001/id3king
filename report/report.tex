\documentclass[11pt]{report}
\usepackage[italian]{babel}
\usepackage[utf8]{inputenc}
\usepackage[T1]{fontenc}
\usepackage[nomarginpar]{geometry}
\usepackage{graphicx}
\graphicspath{ {images/} }

\usepackage[pdftex,
			pdfauthor={Eugenio Severi, Stefano Belli},
			pdftitle={Progetto ASW - id3king-js},
			pdfsubject={Relazione progetto Applicazioni e Servizi Web},
			pdfkeywords={id3king id3king-js ASW},
			pdfproducer={Latex with hyperref},
			pdfcreator={pdflatex}]{hyperref}
\pagestyle{plain}

\begin{document}
\title{Progetto di Applicazioni e Servizi Web\\id3king-js}
\author{Eugenio Severi, Stefano Belli}
\date{A.A. 2016-2017}
\begin{titlepage}
	\maketitle
\end{titlepage}

\setcounter{chapter}{1}
\section{Introduzione}
L'escursionismo è un'attività motoria nel quale viene percorso un tragitto su sentieri in territori tipicamente agevoli, a scopo ricreativo o di studio.
\\Richiedendo uno sforzo fisico variabile in base ai percorsi scelti e molto modulabile nel grado di difficoltà, è una pratica molto diffusa nel mondo e nel nostro paese da persone di tutte le fasce d'età.
\\Gli itinerari solitamente sono mappati su apposite cartine topografiche e il loro stato di percorrenza viene mantenuto costantemente da alcuni enti preposti come il \textit{CAI} (Club Alpino Italiano).
\\Tuttavia, l'escursionismo richiede molta attenzione nella sua pianificazione, in quanto è facile sbagliare la stima della lunghezza della camminata o il suo dislivello:
molti amatori si ritrovano, loro malgrado, a constatare le difficoltà di un percorso soltanto una volta sul luogo, magari a metà del tragitto.
\\\\Fortunatamente, alcuni gruppi di appassionati redigono e pubblicano documentazione sulle escursioni da loro intraprese: il più famoso di essi nella zona emiliano-romagnola è \textit{id3king}.
Il loro sito (\url{www.id3king.it}), infatti, viene aggiornato regolarmente con nuovi dati, inseriti spesso su base settimanale.
Per ogni escursione presente è possibile ottenerne il tracciato GPS, la tabella dei toponimi e una sezione della cartina del percorso, oltre a molte altre indicazioni.
\\Siccome trovare informazioni così dettagliate sui percorsi escursionistici risulta ostico e il loro sito è particolarmente prolifico (allo stato attuale, presenta oltre 890 tragitti documentati nell'arco di diciotto anni!), è evidente che il lavoro del gruppo sia di enorme rilevanza per tutti gli appassionati di questa attività.
\\Tuttavia, allo stato attuale, \textit{id3king.it} non consente di filtrare a piacimento le escursioni inserite, così come ordinarle secondo un altro criterio oltre a quello della data di inserimento crescente; queste mancanze rendono piuttosto complicato trovare rapidamente quelle adatte alle proprie esigenze.
\\\\Alla luce di ciò, è nato il progetto \textbf{id3king-js}: una volta ottenuto il consenso da parte degli amministratori del sito, si è deciso di progettare e sviluppare una web app che esegua lo scraping dell'originale e proponga i dati presenti in maniera maggiormente fruibile e funzionale, ad esempio permettendo una ricerca basata su parametri e la creazione di profili utente per memorizzare dati di interesse.
\\In particolare dovrà essere possibile filtrare e riordinare gli itinerari del sito per consentire una più veloce individuazione di un percorso in base a una certa esigenza.
Ad esempio, un utente potrebbe essere alla ricerca di un percorso non troppo impegnativo: potrebbe pensare di filtrare tutti gli itinerari con lunghezza e dislivello inferiori a un certo quantitativo e magari situati nella zona del lago di Ridracoli (FC).
\\Le funzionalità offerte dall'applicazione possono essere riassunte con il diagramma dei casi d'uso in figura \ref{use_case_diagram}.
\begin{figure}
	\centering
	\includegraphics[scale=0.5]{use_case_diagram}
	\caption{Diagramma dei casi d'uso \label{use_case_diagram}}
\end{figure}
Il sistema dovrà rispettare requisiti di estensibilità, reattività, sicurezza e facilità d'uso per l'utente.
\pagebreak

\section{Architettura}
Per raggiungere gli obbiettivi preposti, sarà necessario realizzare un'opportuna architettura server, un back-end e un front-end in grado di gestire le sessioni utente e restituire i percorsi presenti sul sito; pertanto, l'architettura del sistema è basata su una soluzione client-server.
\\Il server, inoltre, si occuperà di effettuare periodicamente lo scraping del sito: i dati estratti verranno riversati in un apposito database relazionale da cui si potrà attingere per ogni richiesta degli utenti.
\\Riepilogando, l'architettura sarà composta dai seguenti componenti:
\begin{itemize}
	\item un insieme di client che ottengono e utilizzano a piacimento i dati degli itinerari, attraverso un'interfaccia web semplice e intuitiva;
	\item un server web che risponde alle esigenze dei client fornendo loro itinerari e altre funzionalità ad alto livello;
	\item un database relazionale in cui mantenere i dati degli utenti e gli itinerari ottenuti tramite scraping.
\end{itemize}
Adottando una architettura \textit{RESTful}, le comunicazioni I/O dell'applicazione sono completamente \textit{stateless}, consentendo di implementare facilmente un'architettura scalabile a livelli con nodi di commutazione intermedi.
\\Inoltre, essendo le comunicazioni basate su HTTP POST e GET, la portabilità dell'applicazione è totale in quanto universalmente compatibile.
\\In base ai requisiti precedentemente definiti, come framework di sviluppo si è valutato di usare \textit{Node.js}, che prevede l'utilizzo di tecnologie web sia client-side che server-side e consente di utilizzare un modello di comunicazione event-driven, particolarmente utile in un'applicazione di rete.
\\Il pattern \textit{MVC} (Model View Controller) è stato utilizzato per rendere più gestibili, modulari e riutilizzabili le singole componenti.
\\Ai fini di questo progetto si è scelto di utilizzare un unico server cloud di tipo \textit{IaaS} (Infrastructure as a Service), fornito da un provider, sul quale saranno installati tutti i servizi necessari al funzionamento dell'applicazione.
Sono tuttavia possibili configurazioni ridondanti, sia per il server web, sia per il DBMS, al fine di migliorare prestazioni, robustezza, affidabilità e tolleranza ai guasti.
\\In figura \ref{deployment_diagram} è presente uno schema di deployment di base in cui front-end, back-end e database risiedono sullo stesso server.
\begin{figure}
	\centering
	\includegraphics[scale=0.45]{DeploymentDiagram}
	\caption{Diagramma di deployment \label{deployment_diagram}}
\end{figure}
\pagebreak

\section{Sicurezza}
Data la natura di rete dell'applicazione e poiché verranno trattati alcuni dati personali degli utenti (come ad esempio le password), è stato ritenuto opportuno considerare la sicurezza già in fase di progettazione, onde evitare lacune successive difficili da individuare e correggere.
\begin{itemize}
	\item Tutte le comunicazioni di rete sono crittografate e autenticate con SSL/TLS per prevenire intercettazioni e attacchi \textit{man-in-the-middle}; ciò è stato reso possibile grazie alla grande flessibilità del framework Hapi, che consente di impostare semplicemente comunicazioni crittografate client-server.
	Pertanto, tutte le comunicazioni client-server avvengono tramite HTTPS e si possono ritenere quindi relativamente sicure.
	\item Tutte le credenziali degli utenti vengono memorizzate sotto forma di hash con \textit{salt}, grazie alla libreria per NodeJS \textit{bcrypt}, che utilizza l'omonima funzione di hashing la quale, a differenza delle più popolari \textit{SHA*}, adotta un approccio \textit{slow hash}, molto più adatto in questi contesti di sicurezza (a questo proposito, vedere il paper \href{http://worldcomp-proceedings.com/proc/p2016/ICW3865.pdf}{"Implementation and Performance Analysis of PBKDF2, Bcrypt, Scrypt Algorithm"}).
	Pertanto, considerando che le password rielaborate arrivano a sessanta caratteri, si può affermare che la sicurezza delle password salvate sul database sia più che resistente agli attuali metodi di cracking.
	\item Tutti gli input utente vengono validati per impedire lo sfruttamento di attacchi come \textit{SQL injection} e \textit{buffer overflow}, tramite la libreria di interfacciamento col DB, \textit{mysql}. Inoltre, non sono consentite query SQL multiple per una sola richiesta.
	\item L'intero codice sorgente dell'applicazione è disponibile (e aperto a modifiche e miglioramenti) su \textit{Github}.
	Perciò, eventuali \textit{exploit} nel codice sarebbero trasparenti alla comunità e più facilmente segnalabili e correggibili.
\end{itemize}
\pagebreak

\section{Back-end}
Il back-end è implementato in Javascript ES6 su NodeJS, utilizzando il framework \textit{Hapi}.
\\Si è deciso di utilizzare un design modulare, in modo da rendere i componenti maggiormente riutilizzabili e più semplici da testare.
Un controller si occupa di interfacciarsi con i client, fungendo da mediatore con le funzioni più a basso livello, delegate ad apposite classi:
\begin{itemize}
	\item \textit{dbHandler}, che si occupa della comunicazione con il database e fornisce interfacce ad alto livello per eseguire operazioni complesse;
	\item \textit{scraper}, che scansiona il sito \textit{id3king.it} alla ricerca di nuove informazioni da normalizzare ed aggiungere al database.
\end{itemize}
Durante lo scraping vengono eseguite numerose manipolazioni dei dati grezzi: sul sito originale sono inseriti in HTML come stringhe semplici e quindi non normalizzati.
Questa è una condizione necessaria per l'inserimento nel database e per consentire il filtraggio dei percorsi in base ai loro attributi.
\subsection{Diagrammi delle classi e di sequenza}
Nel diagramma delle classi (figura \ref{class_diagram_backend}) è rappresentata la struttura del back-end.
\begin{figure}
	\centering
	\includegraphics[scale=0.45]{ClassDiagram_Backend}
	\caption{Diagramma delle classi del backend \label{class_diagram_backend}}
\end{figure}
\\Nel diagramma di sequenza (figura \ref{sequence_diagram}) è esplicitato il funzionamento di una generica chiamata proveniente da un client che necessita di accedere ad informazioni presenti nel database.
\begin{figure}
	\centering
	\includegraphics[scale=0.45]{SequenceDiagram}
	\caption{Diagramma di sequenza \label{sequence_diagram}}
\end{figure}
\subsection{Dati e database}
Il DBMS scelto per l'implementazione è \textit{MariaDB}.
\\La tabella principale è \textit{Percorso} che, insieme a \textit{Localita}, contiene i dati ottenuti tramite scraping.
Gli amministratori possono opzionalmente modificare manualmente tali dati qualora le informazioni riportate non fossero corrette: questo può accadere poiché i dati originali non sono normalizzati.
Di conseguenza, potrebbero occasionalmente presentarsi dati errati o mal formattati.
Anche per questo motivo, si è scelto di popolare il contenuto di queste tabelle in maniera non distruttiva durante lo scraping automatico.
\\Ogni volta che i dettagli di uno specifico percorso vengono visualizzati da un utente, il relativo contatore viene incrementato: è possibile sapere quali sono gli itinerari più ricercati.
\\Altro fulcro del database è la tabella \textit{Utenti}, che memorizza i dati dei fruitori del servizio e funge da identificatore nelle altre tabelle contenenti le preferenze degli utenti:
\begin{itemize}
	\item \textit{ItinerarioPreferito}, che consente ad ogni utente di salvare una serie di percorsi per poterli ritrovare facilmente in seguito;
	\item \textit{Login}, che memorizza per ogni sessione un token generato casualmente che scade dopo un periodo di tempo configurabile (utile per consentire sessioni multiple);
	\item \textit{Ricerca}, che consente ad ogni utente di ricordare una serie di parametri di ricerca in modo da non doverli reimpostare ad ogni accesso.
\end{itemize}
Concludono lo schema le tabelle \textit{Difficolta} e \textit{Periodo}, necessarie affinché il database sia in terza forma normale.
\\In figura \ref{db_schema} è presente lo schema relazionale.
\begin{figure}
	\centering
	\includegraphics[scale=0.45]{DB_schema}
	\caption{Schema relazionale del database \label{db_schema}}
\end{figure}
\subsection{File di configurazione}
Tramite un file di configurazione (\textit{config.json}) è possibile personalizzare alcuni aspetti del funzionamento dell'applicazione:
\begin{itemize}
	\item la frequenza di scraping automatico (tramite \textit{cron});
	\item i parametri di connessione al DBMS;
	\item la porta su cui il server web deve restare in ascolto;
	\item i requisiti di sicurezza minimi per le credenziali degli utenti;
	\item i file del certificato e della chiave privata per SSL/TLS.
\end{itemize}
\pagebreak

\section{Front-end}
Considerando la grande quantità di dati da visualizzare, la necessità di effettuare richieste client-server e la creazione di un architettura ordinata e ben supportata, gli autori di questo progetto hanno utilizzato \textit{Angular4+} come principale framework di sviluppo.
\\La struttura dell'applicazione segue scrupolosamente la \textit{style guide} ufficiale di Angular, ad esempio nominando i file con la logica "nome-file.tipo.ts" e seguendo il \textit{single responsibility principle}.
\\\\I componenti del progetto sono stati suddivisi in maniera gerarchica, in base alla loro funzionalità:
\begin{itemize}
	\item \textit{views/main}: contiene la struttura principale della tabella, i template dei filtri da applicare ai dati, il modale di creazione filtro e il codice del dettaglio di un itinerario;
	\item \textit{views/header}: definisce la struttura dell'header dell'applicazione e tutta la gestione di login/registrazione e il relativo modale.
\end{itemize}
Tutta la logica di business è stata inserita all'interno dei singoli \textit{services} creati appositamente a tale scopo; essendo presente soltanto un modulo Angular, tutti i servizi utilizzati possono essere considerati \textit{singleton}, in quanto aggiunti nel \textit{providers} del modulo.
\\I \textit{services} progettati possono essere così riassunti:
\begin{itemize}
	\item \textit{LoginService}: fornisce alcuni metodi per effettuare il login e la registrazione dell'utente all'applicazione;
	\item \textit{RouteService}: contiene tutte le principali operazioni effettuabili sugli itinerari (come il loro scaricamento, la registrazione tra i preferiti ecc.), astratti attraverso la classe \textit{Route};
	\item \textit{SessionService}: servizio che mantiene e fornisce la sessione utente qualora un utente si sia autenticato sull'applicazione;
	\item \textit{UtilityService}: fornisce alcuni metodi di utilità utilizzati comunemente, come il download o la traduzione di una data nella rispettiva stagione.
\end{itemize}
Infine, come è intuibile dai nomi dei file, si è scelto di adottare \textit{Typescript} come linguaggio di sviluppo dell'applicazione.
\pagebreak
\subsection{Interfaccia utente}
Il principale obiettivo nello sviluppo del front-end è stato quello di raggiungere una sufficiente usabilità per l'utente finale: ricercare un itinerario in una precisa località, di una sufficiente lunghezza e di una certa difficoltà dovrebbe essere il più semplice ed immediato possibile.
\\Per questo motivo, l'interfaccia dell'applicazione è stata sviluppata quasi esclusivamente sulla tabella contenente i dati degli itinerari, occupando tutto lo spazio disponibile della finestra (figura \ref{ui_1}).
Questa caratteristica, oltre alla possibilità di visualizzare e filtrare gli itinerari senza effettuare obbligatoriamente un log-in, permette fin da subito di soddisfare le esigenze dell'utente.
\begin{figure}[h]
	\centering
	\includegraphics[scale=0.35]{ui_1.png}
	\caption{Interfaccia dell'applicazione \label{ui_1}}
\end{figure}
\\Seguendo questa logica, qualora l'utente voglia approfondire le informazioni di un itinerario, è possibile aprirne il dettaglio cliccando sull'apposita freccia a sinistra del campo ID: tale operazione consentirà di visualizzare dati aggiuntivi e di effettuare alcune azioni, come lo scaricamento della mappa del percorso o della traccia GPS, il salvataggio nei preferiti o l'apertura del link originale dell'itinerario.
\\I filtri sui campi degli itinerari sono stati inseriti all'interno degli header della tabella, permettendo così di filtrare comodamente per:
\begin{itemize}
	\item data (per periodo: inverno, primavera, autunno e estate);
	\item durata di percorrenza;
	\item lunghezza percorso;
	\item dislivello cumulato;
	\item difficoltà nominale;
	\item luogo (toponimo principale dell'escursione).
\end{itemize}
Nel lato superiore sinistro dello schermo è stata aggiunta la possibilità di specificare un unico filtro "globale" da applicare agli itinerari.
\\\\Infine, nel lato superiore destro dello schermo, sono stati posizionati dei link per effettuare il login o la registrazione di un utente.
\\Una volta eseguita l'autenticazione, nell'area sarà possibile accedere a funzionalità avanzate come il salvataggio del gruppo filtri corrente o la visualizzazione dei soli percorsi salvati.
\pagebreak
\subsection{Diagramma delle classi}
Di seguito viene mostrato il diagramma delle classi (e delle relative dipendenze) del \textit{MainComponent} e di \textit{RouteDetailComponent}.
\begin{figure}[h]
	\centering
	\includegraphics[scale=0.45]{ClassDiagram_Frontend}
	\caption{Diagramma delle classi di MainComponent e relative dipendenze \label{ClassDiagram_Frontend}}
\end{figure}
\pagebreak

\section{Conclusioni}


\end{document}
\endinput
